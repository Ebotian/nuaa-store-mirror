\documentclass[a4paper]{article}
% 文档类型为 article,可按需改为 book、report 等
\usepackage[margin=1in]{geometry} % 边距与原模板一致
\usepackage{ctex} % 支持中文排版
\usepackage{multirow} % 复用原表格的多行功能
\usepackage{enumitem} % 控制题目编号格式
\usepackage{amsmath, amssymb} % 支持数学公式
\renewcommand{\baselinestretch}{2} % 行距与原模板保持两倍行距

%==================
% 自定义命令,方便后续维护
%==================
\newcommand{\ExamTitle}{南\(\hspace{0.1em}\)京\(\hspace{0.1em}\)航\(\hspace{0.1em}\)空\(\hspace{0.1em}\)航\(\hspace{0.1em}\)天\(\hspace{0.1em}\)大\(\hspace{0.1em}\)学}
\newcommand{\ExamCourse}{《课程名称》考试试题}
\newcommand{\AcademicYear}{二〇二四\(\hspace{0.2em}\)\(\sim\)\(\hspace{0.2em}\)二〇二五学年}
\newcommand{\Semester}{第\uppercase\expandafter{\romannumeral1}学期}
\newcommand{\ExamDate}{2025年6月25日}
\newcommand{\PaperType}{A}
\newcommand{\PaperCode}{2025A01}

\title{\vspace{-3cm}\kaishu \zihao{-1}{\ExamTitle}}
\author{\vspace{-3cm}}
\date{}

\begin{document}
\maketitle
\vspace{-2cm}

%==================
% 考试抬头信息表
%==================
\begin{table}[!htbp]
    \centering
    \begin{tabular}{|c|c|c|c|c|c|c|c|c|c|c|c|}
        \hline
        \multicolumn{12}{|c|}{\multirow{2}*{\(\hspace{1.8em}\)\AcademicYear\(\hspace{0.5em}\)\Semester\(\hspace{0.4em}\)\kaishu \zihao{-3}{\ExamCourse\(\hspace{1.8em}\)}}}\\
        \multicolumn{12}{|c|}{}\\
        \multicolumn{12}{|c|}{考试日期:\(\hspace{0.4em}\)\ExamDate\(\hspace{2em}\)试卷类型:\PaperType\(\hspace{2em}\)试卷代号:\(\hspace{0.4em}\)\PaperCode}\\
        \hline
        \multicolumn{12}{|c|}{\textbf{班号}\(\hspace{7em}\)\textbf{学号}\(\hspace{7em}\)\textbf{姓名}}\\
        \hline
        \textbf{题号}&\textbf{一}&\textbf{二}&\textbf{三}&\textbf{四}&\textbf{五}&\textbf{六}&\textbf{七}&\textbf{八}&\textbf{九}&\textbf{十}&\textbf{总分}\\
        \hline
        \textbf{得分}& & & & & & & & & & & \\
        \hline
    \end{tabular}
\end{table}

\begin{flushleft}

%==================
% 填写说明与公式编写指导
%==================
\textbf{填写说明:}
\begin{itemize}
    \item 请在每道题后预留的横线或空白处书写答案,保持工整。
    \item 选择题建议使用 `\verb|\begin{enumerate}[label=\Alph*.]|` 环境展示选项,答案可在题目右侧标注 `\underline{\hspace{1.5cm}}`。
    \item 主观题空白区可根据答题需要调整 `\verb|\vspace{Xem}|` 的高度。
    \item 若需插入图形,可使用 `\verb|\includegraphics|` 或 `tikz`,但请注意排版整洁。
\end{itemize}

\textbf{公式编写指导:}
\begin{itemize}
    \item 行内公式使用 `\verb|\( ... \)|`,如 \( e^{i\pi}+1=0 \)。
    \item 行间公式使用 `\verb|\[ ... \]|` 或 `\verb|equation|` 环境,例如:
    \[
        \int_a^b f(x)\,\mathrm{d}x = F(b) - F(a).
    \]
    \item 常用命令:分数 `\verb|\frac{a}{b}|`,根号 `\verb|\sqrt[n]{x}|`,求和 `\verb|\sum_{k=1}^{n} a_k|`,积分 `\verb|\int_{\alpha}^{\beta}|`。
    \item 对复数函数或矢量可使用 `\verb|\mathbf{z}|`、`\verb|\vec{v}|` 等命令,确保语义清晰。
\end{itemize}

%==================
% 第一部分:选择题
%==================
\textbf{一、选择题(共\underline{\hspace{1cm}}题,每题\underline{\hspace{1cm}}分,共\underline{\hspace{1cm}}分)}
\begin{enumerate}[label=\arabic*.]
    \item (示例)函数 \( f(z) = |z|^2 \) 在复平面上是解析函数吗?\\
    \textbf{选项:}
    \begin{enumerate}[label=\Alph*.]
        \item 是
        \item 否
        \item 在实轴上解析
        \item 在虚轴上解析
    \end{enumerate}
    答案:\underline{\hspace{1.5cm}}\hfill 分值:\underline{\hspace{1cm}}

    %\item 在此处复制并修改题目内容,注意保持格式一致
\end{enumerate}

\vspace{2em}

%==================
% 第二部分:填空题
%==================
\textbf{二、填空题(共\underline{\hspace{1cm}}题,每题\underline{\hspace{1cm}}分,共\underline{\hspace{1cm}}分)}
\begin{enumerate}[label=\arabic*.]
    \item 例:\( \frac{i}{1+i} + \frac{-1+i}{i} = \)\underline{\hspace{3cm}}
    \item 例:函数 \( z^3 + 8 = 0 \) 的全部根为 \underline{\hspace{5cm}}
    % 根据需要继续添加题目模板
\end{enumerate}

\vspace{3em}

%==================
% 第三部分:大题(综合题)
%==================
\textbf{三、大题(共\underline{\hspace{1cm}}题,总分\underline{\hspace{1cm}}分)}

% 建议每题使用小标题或括号区分
\textbf{第\underline{\hspace{1cm}}题}\quad(题型示例:证明题/计算题/综合题)
\begin{enumerate}[label=(\arabic*)]
    \item 例题:若 \( f(z) = u(x,y) + iv(x,y) \) 为解析函数,证明 \( u_x \) 与 \( u_y \) 为调和函数。
    \item 说明:可根据题目数量复制本环境,修改题干与空白高度。
\end{enumerate}
\vspace{15em}

\textbf{第\underline{\hspace{1cm}}题}\quad(题型示例:计算题)
\begin{enumerate}[label=(\arabic*)]
    \item 示例:计算 \( \oint_C \frac{z^3}{z^2+1}\,\mathrm{d}z \),其中 \( C \) 为正向圆周 \( |z-i| = \frac{1}{2} \)。
\end{enumerate}
\vspace{18em}

% 根据需要复制更多大题模板

\textbf{附加题或说明:}若需设置附加题,请复制上述结构,并在评分表中新增列或在此处备注评分方式。

\end{flushleft}
\end{document}
