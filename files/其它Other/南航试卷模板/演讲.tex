\documentclass[fontset = none]{beamer}
\usetheme{Madrid}
\useoutertheme{smoothbars}
\def\mathfamilydefault{\rmdefault}
\usepackage{fontspec}
\usepackage[fontset=windowsnew]{ctex}
\usepackage{amsmath}
\setmainfont{Times New Roman}

\definecolor{uibblue}{HTML}{262686}
\definecolor{uibgreen}{HTML}{006000}
\author{032150115易博天}
\institute[NUAA]{南京航空航天大学自动化学院\item
\texttt{yiboxiaotian@nuaa.edu.cn}}

\begin{document}
\begin{frame}
    \title{\TeX 简介及使用入门}
    \author{032150115易博天}
    \institute[NUAA]{南京航空航天大学自动化学院\item
        \texttt{yiboxiaotian@nuaa.edu.cn}}
    \date{\today}
    \titlepage
\end{frame}

\section{WHAT\&WHY}
\begin{frame}{\TeX}
    \TeX是由 Donald E. Knuth[2] 编写的计算机语言,用于文章和数学公
    式的排版。1977 年 Knuth 开始编写 \TeX排版系统引擎的时候,是为了探索
    当时正开始进入出版工业的数字印刷设备的潜力。他特别希望能因此扭转
    那种排版质量下降的趋势,使自己写的书和文章免受其害。我们现在使用
    的 \TeX 系统是在 1982 年发布的,1989 年又略作改进,增进了对 8 字节字
    b符和多语言的支持。\TeX以具有优异的稳定性,可以在各种不同类型的计
    算机上运行,以及几乎没有错误而著称。 \TeX 的版本号不断趋近于\(\pi\),现
    在为3.141592
\end{frame}
\begin{frame}{\LaTeX}
    \LaTeX 是一个宏包(就像一个头文件),其目的是使作者能够利用一个预先定义好的专业页
    面设置,从而得以高质量地排版和打印他们的作品。\LaTeX 最早是由 Leslie
    Lamport[1] 编写的,并使用 \TeX 作为其排版系统引擎。\item
    1994 年, Frank Mittelbach 领导的 \LaTeX 3 小组对 \LaTeX 宏包进行了
    更新,作了一些被期望已久的改进,并且将 \LaTeX 2.09 发布以来数年间出
    现的各种不同的补丁重新统一了起来。这个新版本被称作 \LaTeX \(2\epsilon\),以示
    和旧版本相区别。这也是现在普遍在用的版本。
\end{frame}
\begin{frame}
    比喻:出版的第一步是作者将他们的手稿交给出版公司,然后由图书设计者
    来决定整本书的版面形式(包括栏宽、字体、标题前后的间距……)。图
    书设计者会把他的排版说明写进手稿里,一起交给排版者,排版者最后根
    据这些说明完成这本书的排版工作。在这里, \LaTeX就像那个设计者,而 \TeX就像那个排版者\item
    当然也不用太在意这些
\end{frame}
\begin{frame}{WHY}
    \begin{itemize}
        \item 跨平台,免费,开源
        \item 对数学公式最好的显示,清晰准确
        \item 只用关注逻辑结构,不用自己排版,方便省事%%可以从PPT上举例,对比传统从空白开始的PPT
        \item 国际公认的专业排版标准,美观大方
        \item 广泛的西文期刊接收甚或只接收 TeX 格式的投稿
    \end{itemize}
\end{frame}
\begin{frame}{Weakness}
    \begin{itemize}
        \item 需要一定精力学习计算机和英语%%以及逻辑
        \item 图文混合排版能力较弱
        \item 特效少,不适合一些场合%%比如宣传南航美食
        \item 大多数电脑上没有创作环境%%比如在打印店里就改不了
        \item 大多数人不会,不便于交流
    \end{itemize}
\end{frame}
\section{HOW}
\begin{frame}{Preparation}
    由于跨平台免费开源的特性,每个人都可以选择自己的编写环境,这里仅以我个人为例
    \begin{itemize}
        \item 在win10平台上安装texlive2022,以vscode作为前端写tex文件(事实上我把几乎所有编程类文件都集成在vscode里写了)
        \item 安装步骤:1.下载texlive. 2.下载vscode 3.下载vscode插件
        \item texlive下载地址https://tug.org/texlive/acquire-netinstall.html
        \item vscode下载地址https://code.visualstudio.com/download
        \item 可以现场说明一下我安装的插件。事实上这些也不是必要的,判断标准有且只有一个,能跑就行。
    \end{itemize}
\end{frame}
\begin{frame}{Code}
    \TeX引擎能做很多事,这里给出几个典型应用,我将分别举出这些应用中的一例来分析它的不同使用办法
    \begin{itemize}
        \item 论文
        \item 试卷、作业、笔记
        \item 报告、PPT
        \item 简历
        \item 和MD、HTML的联系
    \end{itemize}
\end{frame}
\begin{frame}{Article}
    这里以IEEE给出的一个模板为例
    \\见IEEE文件夹
\end{frame}
\begin{frame}{Examination}
    这里以我自己仿的两份卷子为例
    \\见Study文件夹
\end{frame}
\begin{frame}{PPT}
    大家应该一直有看吧,现在可以重新看看?
    \\来分析一下我自己的编程结构
    \\双击可以到达对应代码位置,实现可视化效果(大概位置)
\end{frame}
\begin{frame}{CV/Resume}
    这里以moderncv给出的一个模板为例
\end{frame}
\begin{frame}{MD/HTML}
    这里以ppt开始的颜色设置和用到数学模式的MD为例
    \\见test2文件夹
    \\三种标记语言里,HTML最通用,但是太繁琐了,MD开放自由,能输入tex语法的公式,又有HTML的花样,tex严谨规范,但是兼容性总是有些问题。所以还是看情况使用吧
    %%但我最喜欢我的纸和笔
\end{frame}
\section{Conclusion\&Thanks}
\begin{frame}{Conclusion}
    \begin{itemize}
        \item \TeX{}很好看,免费开源,值得信赖,适合做认真严谨的工作,而且做的很好。但是别的场合用它就有些力不从心了吧。一个写好的模板会很方便使用,因为只用关心内容,格式什么的,交给tex就好啦。
        \item \TeX{}是一种标记语言,它的源文件有三种语句,分别是命令、注释和数据,文件主要由两部分组成,前言和正文部分,以begin{document}为分界
        \item 它在很多场合都有应用,主要可以通过最开始的documentclass分析,可以通过具体的class了解对应的用法。%%就像CV或者exam类那样
        \item 照着模板分析可以粗略达到可视化的效果。
    \end{itemize}
\end{frame}
\begin{frame}{Thanks}
    \textbf{Thanks!}
\end{frame}
\end{document}